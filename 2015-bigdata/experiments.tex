\section{Applications and Experiments}
\label{sec:experiments}

The proposed \sd~is useful for a wide range of interesting applications as we
will further discuss in Section~\ref{sec:discuss}.  As specific case studies in
this section, we apply it to three different and representative text mining
tasks related to non-native text analysis in a MOOC or any other online learning
scenario. Please note though, that \sd~could be used in virtually any text
mining environment.

{\bf First}, we show that \sd~can be used to {\em search for a transformation}
of a sentence with grammatical errors  into one with no errors by using native
writing as a reference corpus, thus performing {\em grammatical error
correction} as monolingual translation. This application could be a tool that
students use to correct or grade their own writing. {\bf Second}, we show that
the edits found by \sd~for each sentence can be used {\em as new tokens} to
replace the original text for topical analysis using topic models. When applied
to student essays, this would allow course instructors to find groups of similar
essays that share common errors. These clusters can be viewed as a form of {\em
summary of the corpus} and can be used to form teams, pair complementary
students, or allow batch grading. {\bf Third}, we show that the edits found by
\sd~can be used {\em as features} to improve text representation for the {\em
classification task} of native language identification, for which pure
content-based features tend not to be very effective. Once a student's native
language is known, that information could be used as a fluency score with a
confidence level.

Since our goal is to demonstrate the benefit of \sd~in a variety of different
tasks, and due to the space limit, we do not attempt to optimize the performance
for any of these tasks and thus do not report detailed results for parameter
variations.

All experiments and algorithms \ignore{(including \sd)} are open source and
freely available online as part of the toolkit
\textsc{MeTA}\footnote{\url{https://meta-toolkit.org}}. The NUCLE
corpus\footnote{\url{http://www.comp.nus.edu.sg/~nlp/corpora.html}} (for grammar
correction) and the ICNALE
corpus\footnote{\url{http://language.sakura.ne.jp/icnale/download.html}} (for
summarization and classification) are also freely available online. All
experiments were run on a laptop with an eight-threaded processor and eight
gigabytes of memory.

\subsection{Monolingual Translation}

We use the NUCLE corpus~\cite{nucle} to investigate \sd's performance in
correcting grammatical errors. It is evaluated with precision, recall, and $F_1$
score using the same framework and testing and training data as the CoNLL-2013
Shared Task in Grammatical Error Correction~\cite{conll-shared}.

\subsubsection{Experimental Setup}

We used the 1,036 training data sentences to do parameter selection on the four
different edit penalties and maximum step size. Since the runtime of \sd~is
quite fast on the NUCLE corpus training data, we easily applied grid search on
the weights and $k$ (the maximum number of edits), optimizing the $F_1$ score.
The $n$-gram value was fixed at $n=3$, a standard value for sentence fluency
scoring purposes. As the reference corpus, we used 50,000 sentences from the
Wall Street Journal that are part of the Penn Treebank, since this text is a
staple of well-formed English.

The selected edit weights from the training data were 0.0 for substitute and
base penalties, 0.07 for insert, and 0.30 for remove. This shows that the
default \sd~needs to remove fewer words to get better performance, while
inserting slightly less. The selected value of $k$ was 3.
We set $V^{INS}$ to be a short list of function words, since the omission of
these is a common error. We used a modified version of the Porter2
stemmer\footnote{\url{http://snowball.tartarus.org}} for $V^{SUB}(w)$ that
focuses only on reducing plurals and possessives to the same root.

We tested with the designated 345 testing data sentences and used the evaluation
scripts from the shared task. Given a candidate sentence $S$, the predicted corrected
form is a new sentence $S^*$ that has the lowest perplexity (see Section~\ref{sec:weighted}).

\begin{figure}[t]
\begin{center}
    \begin{tabular}{llrrrrr}
        \hline
        \textbf{Weight} & \textbf{LM} & \textbf{No-op} & $P$ & $R$
        & $F_1$ & $t$ \\
        \hline
        No & No & $0.0\%$ & 2.96 & 4.49 & 3.57 & 120s \\
        No & Yes & $0.8\%$ & 3.22 & 4.47 & 3.74 & \textbf{11s} \\
        Yes & No & $25.5\%$ & 18.78 & \textbf{19.40} & 19.09 & 123s \\
        Yes & Yes & $57.4\%$ & \textbf{35.20} & 17.55 & \textbf{23.42} &
        \textbf{11s}\\
        \hline
    \end{tabular}
    \caption{Grammar correction task: the table shows
        whether edit weights are used, whether insertions are done based on
        perplexity, how many final candidate sentences are unchanged (no-ops),
        precision, recall, $F_1$ score, and runtime in seconds. This system
        would place 7$^{th}$ in the CoNLL shared task.}
    \label{fig-gec}
\end{center}
\vskip-20pt
\end{figure}


\subsubsection{Results}

Fig.~\ref{fig-gec} shows the results of \sd~used for grammatical error
correction, including the precision ($P$), recall ($R$), $F_1$ score, the
running time ($t$), and the percentage of sentences that are unchanged
(\emph{No-op}).  We also included results without edit positions selected by the
language model and results without tuned edit weights. Without edit points
selected, edits are performed at every position in the sentence, generating many
more candidates. Without edit weights, each edit type is treated equally with no
distinction between many or few edits in scoring.

Lack of weighted edits and language model insertion is similar to Lee and
Seneff~\cite{lee-seneff-2006}. Of course, the language model is still used to
score the candidates in all cases. As seen in Fig.~\ref{fig-gec}, compared with this baseline, the intelligent
edit points greatly decrease the run time and the learned edit weights contribute
significantly to the performance improvement.

For a more direct comparison, we can look at the results from the CoNLL shared
task where the teams were judged by $F_1$ score. \sd's score of 23.42 would
place it in seventh overall, beating out $65\%$ (eleven) of the other teams. Not
only does our method place fairly in the shared task standings, but \sd~is a
much more general system than its competitors. The other systems specifically
targeted five error types: article/determiner, preposition, noun number, verb
form, and subject-verb agreement. The standard system first classified errors
into one of the five types. Then, a specific module was run on each error type
in order to produce candidates. Finally, the set of candidates were scored, and
results from each of the five modules was combined into the final corrected
sentence.

\sd~has no concept of different error types and doesn't rely on classifiers to
select particular modules to run. Thus, it is a more general solution than
required for the shared task and can be considered \emph{fluency correction}.

\subsection{Corpus Summarization}

Summarizing student essays can give insight into how they are written.
Comparable essays will have similar deviances from fluent English. Does a group
of students make similar errors? Can we target specific problem areas depending
on the group of students we speak to? Or, can we pair students with
complementary strengths and weaknesses?

Topic models such as latent Dirichlet allocation~\cite{lda} are a powerful text
analysis tool. After running a topic modeling algorithm, each document in a
corpus is assigned a distribution over a fixed number of topics. A topic itself
is a distribution over the corpus vocabulary.

We can use the power of topic models to simultaneously cluster and summarize
errors in non-native English essays and will show that the bag-of-edits
representation enabled by \sd~is much more useful than the standard bag-of-words
representation for this task. \ignore{As we will see, unigram (and even bigram)
words will not be sufficient to understand common fluency differences between
documents. Instead, using \sd~edit features as new tokens to replace the
original words in text captures what we'd like to see: is a group of students
confused about article use? A bag-of-words method only shows the presence of a
term being useful; a bag-of-edits representation shows presence, absence, and
substitution. For each cluster (collection of documents with a concentration of
a particular topic), we are delivered the common differences between the essays
and the native English essays. For a semi-supervised method, these native essays
could be a small subset judged acceptable by the instructors. For a completely
unsupervised method, the reference language model could be from a similar corpus
or a previous semester.} For this task, we use the native English essays on a
similar topic as the reference corpus; this enables us to use \sd~to obtain
edits more likely related to the usage of English language by the students,
which are presumably more useful for this application task than bag-of-words
representation.

\subsubsection{Experimental Setup}

We compare \sd~edit tokens with unigram and bigram words using the 2,800
ICNALE essays debating public smoking. We hypothesize that the edit tokens
will be more interpretable than the competing methods.

Each document is treated as a ``bag-of-edits'', where \sd~is run on each
sentence in every document. A small feature vector for a document could be
$$\{insert(the): 3,\: substitute(a\rightarrow an): 1,\: remove(of): 2\}.$$

We run \textsc{MeTA}'s LDA on this feature set, examining the resulting
distributions of edits and topics. Hyperparameters were set to 0.1, encouraging
sparse distributions.

Since the summarization task is unsupervised, we have no clear objective for
parameter selection. Thus, we leave the weights at zero. However, based on the
observed output, the user is free to adjust the penalties in order to perturb
the results in a direction he or she chooses. Perhaps only substitutions are
currently of interest. Due to space constraints, we do not investigate further
than all zeroed weights. We set $k=1$ to get the most likely change to the
original sentence, and set $n=3$.  We set $V^{INS}$ to the same function word
list as the error correction task and used the full Porter2 stemmer for
$V^{SUB}(w)$ since there was no requirement for such precise substitutions. The
LDA inference is run with a maximum of one thousand iterations, though all three
representations converged before this limit.

\begin{figure}[t]
\begin{center}
    \begin{tabular}{lrrrr}
        \hline
        \textbf{Features} & $|V|$ & $DL_{avg}$ & Iteration $t$ & $500t$ \\
        \hline
        Unigram words & 11,580 & 256 & 3.2s & 26.7m \\
        Bigram words & 130,411 & 255 & 5.4s & 45.0m \\
        \sd & 2,079 & 15 & 0.4s & 3.3m \\
        \hline
    \end{tabular}
    \caption{Summarization task: different tokenization methods for 16 topics on
    the ICNALE smoking corpus. Displayed are vocabulary size, average document
length, LDA inference iteration time, and the time for 500 iterations for
comparison.}
    \label{fig-topic-vocab}
\end{center}
\vskip-20pt
\end{figure}

\begin{figure*}[t]
{\small
\begin{center}
    \textbf{Unigram Words}

    \begin{tabular}{|l|l|l|l|l|}
    \hline
    \textbf{topic 1} & \textbf{topic 4} & \textbf{topic 8} & \textbf{topic 12}
    & \textbf{topic 15} \\
    \hline
    cancer & i & the & restaurant & i \\
    lung   & very & of & banned & japan \\
    smokers & my & tobacco & all & ban \\
    disease & he & cigarettes & country & japanese \\
    heart & was & government & agree & a \\
    nicotine & think & quit & reasons & government \\
    cause & don't & increase & in & just \\
    passive & when & decrease & people & think \\
    \hline
    \end{tabular}

    \vspace{.05in}
    \textbf{Bigram Words}

    \begin{tabular}{|l|l|l|l|l|}
    \hline
    \textbf{topic 1} & \textbf{topic 4} & \textbf{topic 8} & \textbf{topic 12}
    & \textbf{topic 15} \\
    \hline
    smoke cigarette & restaurant owners & if you & the media & passive smoker \\
    smoking zone & smoking bans & you are & harmful for & active smoker \\
    can make & bars and & you can & responsibility of & active smokers \\
    sick in & customers would & when you & hotels or & in indonesia \\
    global warming & or non & you smoke & cigarette companies & the active \\
    this policy & ban on & for your & have shown & can disturb \\
    public space & in bars & around you & and teenagers & more dangerous \\
    make many & smoke filled & yourself and & or anything & all restaurant \\
    \hline
    \end{tabular}

    \vspace{.05in}
    \textbf{\sd}

    \begin{tabular}{|l|l|l|l|l|}
    \hline
    \textbf{topic 1} & \textbf{topic 4} & \textbf{topic 8} & \textbf{topic 12}
    & \textbf{topic 15} \\
    \hline
    insert(the) & insert(a) & remove(you) & remove(so) & remove(area) \\
    remove(opinion) & remove(so) & insert(to) & insert(for) & sub(seat$\rightarrow$seats) \\
    remove(cigarettes) & sub(lung$\rightarrow$lungs) & sub(reason$\rightarrow$reasons) & insert(in) & remove(of) \\
    sub(give$\rightarrow$giving) & sub(make$\rightarrow$makes) &
    sub(ban$\rightarrow$banning) & remove(not) & sub(stop$\rightarrow$stopped) \\
    remove(bans) & remove(healthy) & remove(us) & sub(have$\rightarrow$having) & remove(again) \\
    insert(you) & remove(reasons) & remove(person) & remove(nonsmoker) & insert(i)\\
    remove(totally) & remove(as) & insert(are) & remove(that) & remove(all) \\
    sub(cause$\rightarrow$causes) & remove(even) & remove(better) & remove(increasing) & insert(it) \\
    \hline
    \end{tabular}
    \caption{Summarization task: 5 of 16 topics learned from the
    ICNALE smoking corpus with three tokenization methods. The
$n$-gram methods capture writing themes while \sd~captures similar errors.}
    \label{fig-topics}
\end{center}
}
\vskip-20pt
\end{figure*}



\subsubsection{Results}

Fig.~\ref{fig-topic-vocab} compares vocabulary sizes and iteration runtime for
the LDA inference. Since the \sd~edits have much lower dimensionality than the
word vectors, inference is significantly faster, even on this relatively small
dataset. Fig.~\ref{fig-topics} shows a sample of topics learned from the ICNALE
smoking corpus.

We can see the $n$-gram representations capture more content-based themes while
the edit tokens capture syntactic similarities. For unigram words, topic 1
deals with the physiological concerns of smoking. Topic 12 discusses banning
smoking in restaurants, while topic 15 is more nationally-focused. Topic 4 may
be of some use, suggesting an overuse of personal pronouns.

Bigram words have almost the same interpretability as unigram words. Topic 4 is
similar to topic 12 from the unigram model. Each topic is more of a theme,
rather than a collection of grammatical differences. We only see positive essay
tokens in each topic, as opposed to lacking (missing) ones.

On the other hand, the \sd~edits give some insight into the syntactic structure
of the student essays. For example, consider these excerpts from three different
documents: ``\emph{Because it is so bad to mom with baby}'', ``\emph{In
restaurant when people...}'', ``\emph{...go to restaurant to have meal}''. Each
student has article use errors which \texttt{insert(a)} from topic 4 would fix.
The word $a$ would never appear in an $n$-gram topic model because it is absent
in each of these documents. Such results can also be used to retrieval sample sentences
where the errors occurred.

The same three essays also have an overuse of the word \emph{so}, which
\texttt{remove(so)} from topic 4 would make more fluent: \emph{so bad, so
scared, so dead}. In fact, the first essay contains the phrase \emph{so bad}
five times in about fifteen sentences. The third essay contains the sentence
``\emph{The smoke make many people feel so bad.}'' Aside from the \emph{so}
issue as before, there is a subject-verb disagreement between \emph{the smoke}
and \emph{make}. While other essays may correctly use the verb \emph{make},
these particular essays use it in an incorrect way such that
\texttt{sub(make->makes)} is a correction.

\subsection{Machine Learning}

We use the ICNALE native language identification corpus~\cite{icnale} to test
the effectiveness of using the \sd~edits as features to represent text for
classifying English essays based on the native language of the author. This
corpus contains 5,600 total essays on two prompts. We hypothesize that the
\sd~features capture the grammatical differences in writing styles of the eleven
different native backgrounds.

\subsubsection{Experimental Setup}

The same bag-of-edits representation as the summarization task is used as input
for a classifier to predict the native language of the student essay writer. The
Wall Street Journal sentences from the Penn Treebank are used for the
reference language model as they were for the monolingual translation task.

As a baseline, we use a standard unigram words feature representation with
stemming and stop word removal. Additionally, we combine the unigram words
representation with the \sd~features to see if the performance increases
compared to using only one method.

The ICNALE corpus is split in half, based on whether the essay is a smoking
essay or a part-time job essay. We use the part-time job subcorpus as a
development set to do parameter selection on $n$ and $k$, for the $n$-gram
language model and maximum number of edits respectively. Once the parameters
($k=5,n=5$) were chosen, we evaluated with five-fold cross validation on the
smoking testing set. Each fold of the cross validation is used to do an unpaired
$t$-test for statistical significance. For both development and testing, we use
the default SVM classifier that is part of \textsc{MeTA}. \ignore{The unigram
words baseline and feature combination are also part of the same toolkit.}

Since adding edit weights will always decrease the score of candidate sentences,
we set them all to zero for the classification task. We want the learned
\sd~model to have full control over the generated edits that appear as features.
In contrast to the monolingual translation task, we prefer to \emph{minimize}
the number of no-ops, since each edit operation is used as a feature; more
no-ops means less information is represented. The edit weights are easily set if
the user requires, $e.g.$ to ignore a particular operation. Finally, we leave
$V^{INS}$ and $V^{SUB}(w)$ the same as the summarization task.

\begin{figure}[t]
\begin{center}
    \begin{tabular}{lrrll}
        \hline
        \textbf{Features} & $|V|$ & $DL_{avg}$ & $F_1$ & $Acc$. \\
        \hline
        Unigram words & 9,021 & 129 & 80.1 & 81.8 \\
        \sd & 12,279 & 56 & 73.1 & 75.4 \\
        Combined & 21,300 & 185 & \textbf{84.5}$^*$ & \textbf{85.9}$^*$ \\
        \hline
    \end{tabular}
    \caption{Classification task: comparison between the three methods on the
    ICNALE essays. Displayed are vocabulary size, average document length, $F_1$
score, and accuracy. $^*$Combined results are significantly higher with $p <
0.001$. Each experiment completed in less than 10 seconds.}
    \label{fig-nli}
\end{center}
\vskip-20pt
\end{figure}

%\begin{figure*}[t]
\begin{center}
    \begin{tabular}{|l|rrrrrrrrrrr|}
    \hline
         &  CHN  &    ENS &     HKG  &    IDN  &    JPN  &    KOR  &    PAK  & PHL    & SIN    & THA   &  TWN \\
    \hline
    CHN & \textbf{92}    & 0  & 0     & 0    & 1   & 1    & 1     & 1 & 0 & 1 & 3 \\
    ENS & 0     & \textbf{90}    & 1     & 0    & 2     & 2    & 0     & 2    & 1    & 1    & 1 \\
    HKG & 10    & 3     & \textbf{64}    & 1    & 2     & 0    & 1     & 2    & 8    & 5    & 4 \\
    IDN & 2     & 1   & 1   & \textbf{83} & 0     & 1  & 1     & 3  & 1  & 6  & 1 \\
    JPN & 1     & 1   & 0     & 1  & \textbf{94}  & 2 & 0     & 0    & 0 & 0  & 1 \\
    KOR & 5  & 1     & 1  & 1    & 7  & \textbf{76}   & 1  & 1 & 0    & 6    & 1 \\
    PAK & 1   & 0   & 1     & 0    & 0     & 1    & \textbf{94}  & 2  & 0    & 1    & 0 \\
    PHL & 4   & 1     & 0     & 0    & 0     & 2    & 1   & \textbf{84}   & 2  & 5    & 1 \\
    SIN & 1     & 2     & 0   & 1  & 0   & 2  & 1     & 5  & \textbf{87}   & 1    & 0 \\
    THA & 2  & 0  & 0     & 2    & 1  & 3  & 0  & 1  & 0 & \textbf{90}   & 1 \\
    TWN & 12  & 1   & 2   & 0  & 3     & 6    & 2     & 2  & 1    & 5    & \textbf{66} \\
    \hline
    \end{tabular}
    \caption{Classification task: confusion matrix of combined features on the
        ICNALE corpus. Overall accuracy of 85.9\%. Percentages have been rounded
        for readability. Each ($row,column$) index represents the fraction of
        times $row$ is labeled as $column$; thus all rows sum to 100\%.}
    \label{fig-matrix}
\end{center}
\end{figure*}
 % not enough space

\subsubsection{Results}

Fig.~\ref{fig-nli} shows a comparison between the three methods: unigram words
baseline, \sd, and a combination. While unigram words does outperform edit
features in $F_1$ and accuracy, a combination is able to increase both measures
at a significance level of $p < 0.001$. This shows that the syntactic edit
features capture an orthogonal perspective of the student essays compared to
the lexical features as we expected.

%Fig.~\ref{fig-matrix} shows a confusion matrix of the eleven classes using the
%combined features. Each row is a distribution over which class label was chosen
%for the given row name; the diagonal represents a correct categorization. From
%this, we see that Japanese and Pakistani students are confidently modeled.
%Students from Hong Kong and Taiwan and more easily confused with native
%Chinese speakers, which is logical.

The most informative features for some selected classes are shown in
Fig.~\ref{fig-class-features} according to information
gain~\cite{feature-selection}. Information gain is a commonly-used feature
selection metric in the machine learning and information retrieval communities.
It describes the difference in entropy by knowing the presence or absence of a
specific term appearing in a class.

Some features are obvious and not as informative to the human reader: Chinese
and Korean students overuse \emph{China} and \emph{Korea} compared to the
reference language model. Less apparent (yet still useful) edits are the
Chinese students' overuse of \emph{etc}, the Hong Kong students' underuse of
\emph{the}, the Japanese students' mixup between \emph{so} and \emph{be}, and
the Korean students' differentiation between \emph{fair} and \emph{fairly}. We
also notice that the native English-speaking students have \emph{unmodified} as
a main feature, meaning the perplexity-based candidate scoring preferred their
original sentences over edited ones.

There are also a few artifacts of the tokenization method; the sentence marker
\texttt{<s>} appears as a top feature, implying that English and Korean speakers
tend to have shorter sentences, at least compared to the reference model.

\begin{figure*}[t]
\begin{center}
    \begin{tabular}{lllll}
        \hline
        \textbf{CHN} & \textbf{HKG} & \textbf{ENS} & \textbf{JPN} & \textbf{KOR} \\
        \hline
        remove(people's) & remove(hong) & remove($<$s$>$) & remove(seat) &
        remove($<$s$>$) \\
        remove(china)    & remove(kong) & insert(is) & remove(nonsmoking) & remove(korea) \\
        insert(the)  & insert(the)  & remove(bad) & remove(tobacco's) & remove(sterility) \\
        remove(harmony)  & sub(forced$\rightarrow$forcing) &
        insert(a) & sub(so$\rightarrow$be) & insert(or) \\
        sub(people's$\rightarrow$people) & remove(don't) & unmodified & remove(can't) & remove(rice) \\
        remove(etc)      & remove(carcinogenic) & remove(good) & remove(foods) & remove(habit) \\
        insert(such) & sub(affected$\rightarrow$affecting) & insert(such)
        & remove(opinion) & remove(non) \\
        sub(terrible$\rightarrow$terribly) & remove(country) & insert(to)
        & remove(two) & sub(fair$\rightarrow$fairly) \\
        \hline
    \end{tabular}
    \caption{Classification task: edit features selected via information gain
    for 5 of the 11 classes in the ICNALE corpus.}
    \label{fig-class-features}
\end{center}
\vskip-20pt
\end{figure*}


\begin{abstract}
We describe \sd, a novel, general, and efficient edit-based method for
transforming sequences of words given a reference text collection. These
transformations can be used directly or can be employed as features to represent
text data in a wide variety of text mining applications. As case studies, we
apply \sd~to three quite different tasks, including grammatical error
correction, student essay clustering and analysis, and native language
identification, showing its benefit in each case. \sd~is completely general and
can thus be potentially applied to any text data in any natural language. It is
highly efficient, customizable, and able to capture syntactic differences from a
reference text collection at the sentence, document, and subcollection levels.
This enables both a rich translation method and feature representation for many
text mining tasks that deal with word usage and syntax beyond bag-of-words.
\end{abstract}

\begin{IEEEkeywords}
Comparative Text Mining, Monolingual Translation, Corpus Summarization, Text
Categorization
\end{IEEEkeywords}

\documentclass[conference]{IEEEtran}
\usepackage{url}
\usepackage{algpseudocode,algorithm}
\usepackage{./parsetree}
\usepackage{cite}
\usepackage{latexsym}

% options for parsetree.sty
%\ptnodefont{\normalsize\rm}{10pt}{2pt}  % font and strut height/depth: nodes
%\ptleaffont{\normalsize\it}{10pt}{2pt}  % font and strut height/depth: leaves

\begin{document}
\title{Structural Parse Tree Features \\ for Text Representation}

\author{%
    \IEEEauthorblockN{Sean Massung}
    \and
    \IEEEauthorblockN{ChengXiang Zhai}
    \IEEEauthorblockA{%
        \\
        \emph{Department of Computer Science, College of Engineering} \\
        \emph{University of Illinois at Urbana-Champaign} \\
        \texttt{\{massung1,czhai,juliahmr\}@illinois.edu}
    }

    \and
    \IEEEauthorblockN{Julia Hockenmaier}
}

\maketitle

\begin{abstract}
We propose and study novel text representation features created from parse tree
structures. Unlike the traditional parse tree features which include all the
attached syntactic categories to capture linguistic properties of text, the new
features are solely or primarily defined based on the tree structure, and thus
better reflect the pure structural properties of parse trees. We hypothesize
that these new complex structural features capture an orthogonal perspective of
text even compared to advanced syntactic ones. Evaluation based on three
different text categorization tasks (\emph{i.e.}, nationality detection, essay
scoring, and sentiment analysis) shows that the proposed new tree structure
features complement the existing ones to enrich text representation. Experiment
results further show that a combination of the proposed new structure features
with word $n$-grams can improve $F_1$ score and classification accuracy.
\end{abstract}

\section{Introduction}

Text representation is a fundamental issue in many text processing applications
such as information retrieval, text clustering, and text categorization. Text
representation is also critical for generating useful features to be used in
many machine learning algorithms to support natural language processing
applications. In general, we can represent text by a set of extracted features;
the simplest features can be just the words in the text. 

The issue of text representation is complicated because different tasks tend to
require a somewhat different perspective of representation---thus a different
feature set.  For example, while functional words are generally not useful for
topic categorization, they may be useful for the author attribution
categorization task, which may also benefit from features capturing sentence
structures.  It is therefore important to develop a rich set of potential
features that can represent text from different perspectives and to understand
what kind of features are most effective for which tasks. 
 
By far, the most common way to generate features is to segment text into words
and record their $n$-grams; indeed, unigrams are quite common for information
retrieval and text classification applications.

In addition to content features, functional words and syntactic features have
also been considered, notably for tasks such as author attribution or essay
scoring. Complementary with content features, syntactic features can better
reflect the writing style of an article.  For example,  simple syntactic
features such as $n$-grams of part-of-speech tags and unigram function words
were used for authorship attribution in \cite{jasist-stamatatos-2009} and
\cite{jasist-koppel-2009}.  To further capture syntactic structures,
grammatical productions (rewrite rules) were also discussed as potential
features in \cite{allc-baayen-1996}, where the authors showed that rule
frequencies were significantly different across classes and used them as
features in some simple classification tasks. Later work used syntactic tree
features for scoring non-native speech \cite{acl-chen-2011}, authorship
attribution in \cite{acl-raghavan-2010} and \cite{sigir-kim-2011},
deception detection in \cite{acl-feng-2012}, relation extraction in
\cite{naacl-jiang-2007}, and even tree kernel methods in
\cite{lsm-agarwal-2011} and \cite{jis-zhou-2010}.

In this paper, we propose to investigate a new dimension of text representation
based on parse trees with more emphasis on structural representation.
Specifically, we propose to define structural features solely based on
structural properties of a parse tree by ignoring all of the syntactic
categories in the tree. More specifically, we call such new features skeletons
to indicate their emphasis on pure structures rather than rewrite rules. A
skeleton is defined as any subtree of a parse tree without including any
syntactic categories. Compared with syntactic rewrite rules, skeletons can
better capture the structural properties of a whole parse tree. Indeed, an
important advantage of skeletons over regular syntactic features is that they
can capture ``global tree structures" without causing problems of data
sparseness or overfitting. Because of the focus on pure structures, even
relatively large skeletons can be observed multiple times in a reasonably large
set of text articles; in contrast, if we are to attach the syntactic categories,
we would end up having far more specialized features that may not be observed
multiple times in a data set.  We thus hypothesize that skeletons can capture a
new additional dimension of text that cannot be easily captured by either
content features or regular syntactic features, and thus may serve well as
complementary features with the existing ones.

We evaluate the proposed skeleton-based features using three different
categorization tasks that likely would benefit from structural representation of
text: nationality detection, essay grading, and sentiment analysis. We compare
feature combinations of the proposed new features with three common simple
features ($n$-grams of words, part-of-speech tags, and function words). We also
investigate existing tree features (rewrite rules, syntactic categories,
production branching factors, and tree depth), showing that the new
skeleton-based features provide orthogonal information compared to the simpler
features and validating their usefulness for text representation.

\section{Related Work}

Tree structure has been explored before, though not in a text representation
perspective. A treebank described in \cite{coling-black-1996} allows
grammatical parse trees to be browsed based on structure alone, but does not
provide any sort of classification component. In \cite{acl-wang-2007}, the
authors use dependency tree structure in a sentence similarity metric for
textual entailment. A sentence similarity measure could possibly be generalized
to an entire document, though a purely-structural sentence similarity measure
has not been presented before.

Although both these works consider tree skeletons, they are not used as a
feature for text representation, and thus have not been used as features for
classification, clustering, or information retrieval.

In previous studies of features for text representation, the authors only
examined a small subset of feature competitors: in \cite{acl-raghavan-2010},
unigrams, bigrams, and trigrams of words; in \cite{sigir-kim-2011}, unigram
and bigram part-of-speech (POS) tags and bag-of-words function words (FW); in
\cite{acl-feng-2012}, unigram and bigram words and unigram, bigram, and
trigram parts-of-speech.  In addition to a limited comparison set, each paper
only considered one domain; authorship attribution in the first two, and
deception detection in the second. In this paper, we compare these existing
features with the proposed new features on three additional tasks: nationality
detection, essay scoring, and sentiment analysis.

\cite{acl-chen-2011} examines tree features at a much higher level in the
form of nonterminals per sentence, \emph{e.g.} number of noun phrases per
sentence and mean number of prepositional phrases per sentence. The work was
mainly focused on investigating potential features so no classification tasks
were performed.

\cite{sigir-kim-2011} mines discriminative frequent PCFG tree patterns for
each author. Features used were the rewrite rules and a new pattern,
$k$-embedded-edge ($ee$) subtrees: subtrees that share a set of $k$
ancestor-descendant subtrees.  Therefore, a 0-$ee$ subtree would be one
arbitrarily-sized subtree, and a 1-$ee$ subtree would be one subtree and one
descendant subtree anywhere in the parse tree.  This creates an exponential
number of potential patterns, and the authors define algorithms in order to
process this large amount of data before pruning the number of $ee$ trees to be
used as features. In fact, the algorithms were run on a petascale supercomputer,
which justified the implication that their method is quite computationally
intensive. Besides the concern of computational complexity, another concern is
the high susceptibility of the large number of patterns to overfitting.  In
contrast,  the skeleton features proposed  in this paper are efficient to
compute and systematically capture the major structures in a parse tree. 

\cite{naacl-jiang-2007} explores feature extraction from sequence (words),
syntactic (grammatical parse trees), and dependency (dependency parse trees)
subspaces. Features used were $n$-grams for the word sequences, grammar
productions for PCFGs, and dependency paths for the dependency parse trees. They
concluded that adding all these features together versus separately only
slightly increases performance. We suspect that this is because the structural
information encoded in the parse trees is not taken into account.

Grammatical parse tree features have also been explored in classification tasks
as tree kernels in \cite{acl-collins-2002}, \cite{emnlp-kudo-2004}, and
\cite{eacl-moschitti-2006}. Again, none of this previous work takes into account
the structure of the trees themselves, but rather focuses on the syntactic
categories as the main avenue of information. \cite{acl-collins-2002} mainly
focuses on reducing the feature space of ``all subtrees'' for the perceptron
algorithm using rewrite-rule features. \cite{emnlp-kudo-2004} explores a
boosting method over ``subtree stumps'' (common subtree sequences). Finally,
\cite{eacl-moschitti-2006} provides yet another tree kernel method, focusing on
efficient algorithms. They use parse tree substructures, a somewhat larger
feature space than rewrite rules, but still consider only the node labels in
addition to their order.

We evaluate our proposed skeleton features with three applications: nationality
detection, essay scoring, and sentiment analysis. These tasks were previously
studied in many papers, including, \emph{e.g.},
\cite{icassp-kamel-2010},    % nationality detection, very hard to find a paper like we want
\cite{coling-burstein-1998}, \cite{shermis-2003},    % essay scoring
\cite{ftir-pang-2008}, and \cite{esa-tang-2009}    % sentiment analysis
. Our experiments show that the proposed new features can
help  improve performance for all these tasks, and potentially many others.  
 
\section{Existing Text Representation Methods}

We now discuss existing text representation methods and their motivation. For
this section and the next, a parse tree of the sentence ``\emph{They have many
theoretical ideas.}'' is used for examples and given below.

\begin{center}
\begin{parsetree}
  (.S. (.NP. (.PRP. `They')) (.VP. (.VBP. `have')
    (.NP. (.JJ. `many') (.JJ. `theoretical') (.NNS. `ideas')))
    (.$\bullet$. `$\bullet$') )
\end{parsetree}
\end{center}

The parse tree is rooted with $S$, denoting $Sentence$; the sentence is composed
of a noun phrase ($NP$) followed by a verb phrase ($VP$) and period. The leaves
of the tree are the words in the sentence, and the preterminals (the direct
parents of the leaves) are part-of-speech tags.

\subsection{Words}

Words are the simplest, most intuitive textual features. Although not a
grammatical feature such as rewrite rules or function words, we include them
because of their prevalence and accuracy. Besides, we expect words will be a
very useful feature to combine with more advanced methods.

Stemming and stop word removal are optimizations often applied to word features.
Both these methods greatly reduce the feature space, omitting irrelevant words
(stop word removal) and grouping words with the same meaning together, such as
``player" and ``playing" (stemming).

$n$-grams of words and all the following simple features are also collected and
utilized. This is quite a common practice, with $n$-grams of words first
described in \cite{misc-furnkranz-1998}.

\subsection{Part-of-Speech Tags}

Part-of-speech tags are a common grammatical feature. Their small, finite number
lends them to be simple features for a classifier. When expanding to $n$-grams
of part-of-speech tags, their small number also ensures that there are still a
relatively low number of features generated (opposed, mainly, to words).

POS tags are perhaps the syntactic analog of basic words, in that they are
simple and robust. They capture grammar usage at its most basic level. High
accuracy POS taggers ($\geq97\%$) ensure cleanly processed data.

\subsection{Syntactic Categories}

Syntactic category features can be thought of as an extension of POS tags to
parse trees.  This creates a distribution of non-terminal productions over each
class.  The trees are simply traversed, tallying the labels of internal nodes:
S:1, NP:2, VP:1, PRP:1, VBP:1, JJ:2, and NNS:1.

The goal of syntactic categories is to observe phrase structure occurrence at a
level above POS tags and words. This feature is similar to POS tags in that
sense, and it even records them when examining nodes near the leaves of the
tree. We hope this feature does at least as well as unigram POS tags, since they
are a strict subset of this feature.

\subsection{Function Words}

Function words are a well-performing feature for authorship attribution as noted
by \cite{jasist-stamatatos-2009} and \cite{jasist-koppel-2009}. They attempt to
capture nuances in text that remain largely an unconscious byproduct of
individual authors. In our experiments, we see if our 320 function words also
distinguish between nationality, essay grades, or positive or negative
sentiment.

Since the $n$-gram feature generation tools in our toolkit already existed for
POS tags and words, we ran the function words collected from the
text through this part of the system as well, mainly out of curiosity if bigram
function words or higher turned out to be useful.

\subsection{Tree Depth}

Tree depth simply measures the maximum number of productions to travel to reach
a leaf. In the example above, this would be four (including the word terminals).
The depth is tallied for all trees in a label, creating a distribution of
depths.

Since tree depth is approximately proportional to sentence length, it may prove
to be a useful feature since sentence length is often used as a feature for
distinguishing texts. Depth was explored in \cite{acl-chen-2011} and found to be
positively correlated to scores in \emph{spoken} exams.

\subsection{Production Branch Factors}

This creates a histogram of ``branching factors" at each non-terminal.  For
example, the $S$ production in the example has three productions, therefore
contributing one to the branch count value ``3". Remember, we ignore the
terminal productions. In this feature case, it would inflate the counts for a
branch factor of 1 anyway.

Statistics for the example parse are as follows: one count of one branch for
NP$\rightarrow$PRP; one count of two branches for VP$\rightarrow$VBP, NP; and
two counts for three branches in S$\rightarrow\dots$ and NP$\rightarrow\dots$

Counting branches attempts to capture the complexity of a production (and the
overall sentence). Generating more labels usually results in a longer, more
involved sentence, perhaps indicative of specific classes.

\subsection{Rewrite Rules}

This method tallies subtrees from each sentence's parse and was one of the tree
features in \cite{sigir-kim-2011} and others. The following subtrees
would be recorded from the example sentence \emph{``They have many theoretical
ideas."}:

\begin{center}
\begin{tabular}{ll}
    \begin{parsetree}
        (.S..NP..VP..$\bullet$.)
    \end{parsetree} &
    \begin{parsetree}
        (.NP..PRP.)
    \end{parsetree} \\
     &  \\ % add space
    \begin{parsetree}
        (.VP..VBP..NP.)
    \end{parsetree} &
    \begin{parsetree}
        (.NP..JJ..JJ..NNS.)
    \end{parsetree} \\
\end{tabular}
\end{center}

This process is repeated for all sentences in all texts belonging to a given
class, so each class has a distribution of these subtrees. It can be thought of
as a ``bag-of-trees" method.

This feature is desirable, as particular parse trees could be common for any
particular category. For example, in age detection, more complicated tree
structures could be scarce for younger writers.  Similarly, authors whose native
language is not English may only select sentence structures from a relatively
small learned collection, or repeat similar practiced patterns.

\section{Novel Skeleton-Based Features}

\subsection{Skeletons}

The skeleton method is a novel procedure that recursively descends into
subtrees, recording the internal structure with disregard to internal node
labels. This attempts to capture the flow or phrasal structure of sentences
while being agnostic to actual labels.

The simple \textsc{CountSkeletons} function is described below. \textsc{Skeleton}
returns the skeletal structure of the tree rooted at the parameter.

\begin{algorithm}
\caption{Counting different skeletons in a parse tree}
\begin{algorithmic}
    \Procedure{CountSkeletons}{$T$}
    \State $token$ $\leftarrow$ \textsc{Skeleton}($T$)
    \State \textsc{incrementCount}($token$)
    \For{each subtree $t\in T$}
        \State \textsc{CountSkeletons}($t$)
    \EndFor
    \EndProcedure
\end{algorithmic}
\end{algorithm}

As with the previous feature, simple frequency counts are kept for each tree
skeleton in each sentence in the entire class data set as indicated by the
function \textsc{IncrementCount}. The skeleton structure representations can be
recorded as sets of parenthesis: \texttt{((())(()(()))())}.

For example, here are the skeletons generated from the sentence \emph{They have
many theoretical ideas}:

\begin{center}
\begin{tabular}{ll}
    \begin{parsetree}
        (.x.
          (.x..x.)
          (.x..x.
          (.x..x..x..x.)).x.)
    \end{parsetree} &
    \begin{parsetree}
        (.x..x.(.x..x..x..x.))
    \end{parsetree} \\
    \begin{parsetree}
        (.x..x.)
    \end{parsetree} &
    \begin{parsetree}
        (.x..x..x..x.)
    \end{parsetree} \\
\end{tabular}
\end{center}

\subsection{Annotated Skeletons}

Annotated skeletons are a compromise between rewrite rules and raw skeletons.
They hope to form a middle ground between the specificity of rewrite rules and
the generality of skeletons.  The algorithm to generate these features is almost
exactly the same as skeleton's, except the topmost internal node's label is
retained (the annotation).

Again, the pseudocode for obtaining annotated skeletons is given below. The
function \textsc{Category} returns the syntactic category of its parameter,
\emph{e.g.} $VP$ or $CC$.

\begin{algorithm}
\caption{Counting annotated skeletons in a parse tree}
\begin{algorithmic}
    \Procedure{CountAnnotatedSkeletons}{$T$}
    \State $token$ $\leftarrow$ \textsc{Category}($T$) + \textsc{Skeleton}($T$)
    \State \textsc{incrementCount}($token$)
    \For{each subtree $t\in T$}
        \State \textsc{CountAnnotatedSkeletons}($t$)
    \EndFor
    \EndProcedure
\end{algorithmic}
\end{algorithm}

An annotated skeleton feature could be as follows: \texttt{(S(())(()(()))())}.
Given the example sentence, each subtree above would be given a frequency count
of one.

\begin{center}
\begin{tabular}{ll}
    \begin{parsetree}
        (.S.
          (.x..x.)
          (.x..x.
          (.x..x..x..x.)).x.)
    \end{parsetree} &
    \begin{parsetree}
        (.VP..x.
        (.x..x..x..x.))
    \end{parsetree} \\
    \begin{parsetree}
        (.NP..x.)
    \end{parsetree} &
    \begin{parsetree}
        (.NP..x..x..x.)
    \end{parsetree} \\
\end{tabular}
\end{center}

A key difference between the skeleton-based features and the existing rewriting
rules features is that the skeleton-based features emphasize pure structural
properties by intentionally ignoring the syntactic labels.  Thus they can
represent text data from an orthogonal perspective to what existing features can
capture. Furthermore, an advantage of such features as compared with rewrite
rules is that they are generally more frequent, therefore less likely to suffer
from data sparseness.

From another perspective, we may also view a skeleton feature as a cluster of
syntactically annotated tree structures that share the same underlying
structure. An annotated skeleton is simply a more restricted cluster with the
root node fixed to a phrasal category.

\section{Evaluation}

\subsection{Data Sets}

We evaluate the proposed features using three different text categorization
tasks that likely benefit from using structural features.

The \cite{ceeaus} data set consists of 1008 essays written in English by
native Chinese, Japanese, and English students. Essays were classified by their
writers' native language. In attempts to keep content uniform, each essay is a
response to one of two writing prompts: 1) \emph{It is important for college
students to have a part-time job} or 2) \emph{Smoking should be completely
banned at all restaurants in the country}. Categorizing text based on assumed
nationality would be a useful way to rate one's mastery of a second language. It
would also aid in authorship profiling when combined with other methods trained
on age and gender.

The \cite{kaggle-essay} data set is 10,686 scored student essays on a range
of 0 to 12. Essays were relatively short, all between 150 and 550 words. These
essays were originally used as data for a contest in essay scoring. Scores for
the essays are an average of three human graders' scores in an attempt to
portray the most accurate human judgement.

The data set from \cite{acl-maas-2011} consists of 50,000 movie reviews from
the International Movie Database, classified as either positive or negative. All
movie reviews are scored out of 10, but only clearly negative (score $\leq 4$)
or clearly positive (score $\geq 7$) are included in the data set for
data polarization.

\begin{figure*}[ht!]
\begin{center}
%{\small

\begin{tabular}{ccc}

    \textbf{Nationality Detection} &
    \textbf{Essay Scoring} &
    \textbf{Sentiment Analysis} \\

    % nationality
    \begin{tabular}{|lccc|}
    \hline
    \emph{Method} & $n_{best}$ & $F_1$ & $A$ \\
    \hline
    \hline
    Word    & 1 & .827 & .916 \\
    POS     & 2 & .810 & .902 \\
    FW      & 2 & .728 & .876 \\
    \hline
    \hline
    RR      &   & .778 & .844 \\
    Skel    &   & .510 & .806 \\
    ASkel   &   & .721 & .870 \\
    SC      &   & .703 & .844 \\
    Branch  &   & .508 & .786 \\
    Depth   &   & .436 & .774 \\
    \hline
    \hline
    \textbf{ASkel + Word} & \textbf{1}  & \textbf{.885} & \textbf{.942} \\
    SC + Word    & 1  & .867 & .936 \\
    RR + Word    & 1  & .854 & .924 \\
    ASkel + POS  & 2  & .811 & .900 \\
    RR + POS     & 2  & .810 & .896 \\
    Skel + Word  & 1  & .809 & .912 \\
    ASkel + FW   & 2  & .799 & .904 \\
    RR + FW      & 2  & .795 & .898 \\
    \hline
    \end{tabular}
    &
    % essay
    \begin{tabular}{|lcc|}
    \hline
    \emph{Method} & $n_{best}$ & $\kappa$ \\
    \hline
    \hline
    \textbf{Word}   & \textbf{2} & \textbf{.889} \\
    POS    & 2 & .765 \\
    FW     & 1 & .845 \\
    \hline
    \hline
    RR     &   & .702 \\
    Skel   &   & .356 \\
    ASkel  &   & .658 \\
    SC     &   & .431 \\
    Branch &   & -.087 \\
    Depth  &   & -.098 \\
    \hline
    \hline
    SC + Word    & 2 & .834 \\
    ASkel + FW   & 1 & .822 \\
    RR + Word    & 2 & .807 \\
    ASkel + Word & 2 & .791 \\
    RR + FW      & 1 & .786 \\
    ASkel + POS  & 2 & .782 \\
    Skel + POS   & 2 & .768 \\
    RR + POS     & 2 & .761 \\
    \hline
    \end{tabular}
    &
    % sentiment
    \begin{tabular}{|lccc|}
    \hline
    \emph{Method} & $n_{best}$ & $F_1$ & $A$ \\
    \hline
    \hline
    Word   & 1 & .820 & .820 \\
    POS    & 3 & .662 & .662 \\
    FW     & 1 & .687 & .687 \\
    \hline
    \hline
    RR     &   & .650 & .650 \\
    Skel   &   & .556 & .557 \\
    ASkel  &   & .654 & .654 \\
    SC     &   & .555 & .568 \\
    Branch &   & .493 & .499 \\
    Depth  &   & .533 & .535 \\
    \hline
    \hline
    \textbf{Skel + Word}  & \textbf{1 } & \textbf{.828} & \textbf{.828} \\
    RR + Word    & 1  & .824 & .824 \\
    ASkel + Word & 1  & .824 & .824 \\
    SC + Word    & 1  & .822 & .822 \\
    ASkel + FW   & 1  & .704 & .704 \\
    RR + FW      & 1  & .686 & .686 \\
    ASkel + POS  & 3  & .685 & .685 \\
    Skel + FW    & 1  & .682 & .682 \\
    \hline
    \end{tabular}

\end{tabular}
\caption{
{\normalsize Comparison of single and combined features across data sets.  RR, Skel,
    ASkel, and SC refer to Rewrite Rules, Skeleton, Annotated Skeleton, and
    Syntactic Category respectively.
}}
\label{fig:three-datasets}
%}

\end{center}
\end{figure*}



\subsection{Measures}

To assess classification accuracy for nationality detection and sentiment
analysis, we employ the commonly used information retrieval measurements $F_1$
score and accuracy \cite{rocai-caruana-2004}.

The essay data set is evaluated differently. As in the original contest,
performance is calculated with the quadratic weighted $\kappa$ metric, described
in \cite{pb-cohen-1968}.  In short, the $\kappa$ metric measures agreement
between two raters using a fixed scale, where usually $\kappa\in [0.0, 1.0]$,
with 0.0 indicating random agreement and 1.0 indicating exact agreement. For
very poor features, it is possible that the score drops below 0.0. The contest's
own evaluation script was run on our output.

\subsection{Experiment Design}

The main questions we strive to answer are $Q_1)$ Are the new features
orthogonal to the existing ones? and $Q_2)$ Can we combine the new features with
old ones to improve accuracy?

In order to conduct fair experiments, we created a modular testing framework
that easily allows us to exchange features and data sets. Source texts were
preprocessed with the Stanford parser \cite{stanford-parser} and
part-of-speech tagger \cite{stanford-pos}, then features were generated
based on the preprocessed data. The feature files are then passed to liblinear
\cite{liblinear} (an SVM library), where it learns a classifier and performs
five-fold cross-validation to evaluate the results. We used the parameter
\texttt{-s 1} for all runs, referring to \emph{L2-regularized L2-loss support
vector classification (dual)}. This configuration has $C=1,B=-1,\epsilon =
0.1$.

In reference to authorship attribution, \cite{jasist-stamatatos-2009} notes
that the ``SVM model is able to avoid overfitting problems even when several
thousands of features are used and is considered one of the best solutions of
current technology". Hence we chose to use SVMs as our classification method,
though of course any classifier could be used.

For word features, 433 stop words based on the Lemur toolkit's \cite{lemur}
stop word list are removed.  Then, the words are stemmed according to the
Porter2 stemmer \cite{porter2}.

We compare the three baseline features (words, POS tags, function
words) with the tree features (rewrite rules, skeletons, annotated
skeletons, syntactic categories, tree branch factors, and tree depth).

We partition each data set into two parts; on the first, we perform parameter
selection via five-fold cross-validation to find the best $n$ for words,
part-of-speech tags, and function words. Then, we select the best-performing $n$
from this set and run it, the tree features, and tree features + best $n$-grams
on the second part, again with five-fold cross-validation.

Software used to run all experiments presented in this paper is open-source and
freely available online.\footnote{https://bitbucket.org/smassung/meta}

\subsection{Results}

Figure~\ref{fig:three-datasets} shows the evaluation results on the three data
sets. Each column is split into three parts; the top records performance for the
best-performing single methods for all $n$-grams. The middle section shows
tree-based method results, and the bottom section shows the best-performing
combined methods.

On the nationality data set, we see that of the single methods, word unigrams
performed the best with an $F_1$ score of .827, where the best tree feature (RR)
had .778. Combining features showed annotated skeletons and unigram words proved
most effective ($F_1 = .885$). 

The contest that originated the essay data set ended before this work was begun;
the first place finisher ended up with a score of $\kappa =.8141$, but this
score is not directly comparable to our results since we believe the contest
scored entries based on a withheld testing set. For this task, word features
performed the best, with none of the structural features being beneficial. In
fact, both tree depth and tree branch factors did so poorly that their scores
ended up negative.  This shows that the effectiveness of features clearly
depends on the task. Perhaps the essay scoring is more dependent on content
rather than writing style.

Similarly to the nationality detection experiment, word unigrams performed the
best in the sentiment analysis task ($F_1 = .820$). For combined features,
skeleton and unigram words performed the best with ($F_1 = .828$).
\cite{acl-maas-2011} uses this data set, testing with two folds. They achieved
$A=88.89$. \cite{acl-wang-2012} also cites using this data set, with $A = 91.22$
on what we assume to be two folds. We note that our results are significantly
less, due to using half the data set for $n$-gram parameter selection before
running the experiments on the other half.

\section{Discussion}

\subsection{The Best Features to Combine}

Without a doubt, adding additional tree features enhances performance. But which
tree features are most effective when combined? We hypothesize that the skeletal
features (skeleton and annotated skeleton) result in the best combined
performance.

In order to further investigate these hypotheses,
Figure~\ref{fig:multiple-features} shows the relative gain obtained by adding
the tree features skeleton, annotated skeleton, rewrite rules, and syntactic
categories to the four original methods.

\begin{figure}[t!]
%{\small
\begin{center}

    \textbf{Nationality Detection $F_1$ Gain} \\
    \begin{tabular}{|lrrrrr|}
    \hline
    \emph{Method} & $n$ & Skel & ASkel & RR & SC \\
    \hline
    Word & 1 &  -.018  & .058 & .027 & .040 \\
    POS  & 2 &  -.045  & .001 & .000 & -.025 \\
    FW   & 2 &  -.074  & .071 & .067 & .024 \\
    \hline
    $\bar{x}$ && -.046  & \textbf{.043} & .031 & .013 \\
    $\sigma$  && .028 & .037 & .034 & .033 \\
    \hline
    \end{tabular} \\ ~ \\

    \textbf{Essay Scoring $\kappa$ Gain} \\
    \begin{tabular}{|lrrrrr|}
    \hline
    \emph{Method} & $n$ & Skel & ASkel & RR & SC \\
    \hline
    Word & 2 &  -.140  & -.098 & -.082 & -.055 \\
    POS  & 2 &  .003  & .017 & -.004 & -.099 \\
    FW   & 1 &  -.132  & -0.23 & -.077 & -.145 \\
    \hline
    $\bar{x}$ && -.090  & -.104 & -.054 & -.100 \\
    $\sigma$  && .080 & .123 & .044 & .045 \\
    \hline
    \end{tabular} \\ ~ \\
    
    \textbf{Sentiment Analysis $F_1$ Gain} \\
    \begin{tabular}{|lrrrrr|}
    \hline
    \emph{Method} & $n$ & Skel & ASkel & RR & SC \\
    \hline
    Word & 1 &  .008  & .004 & .004 & .002 \\
    POS  & 3 &  .015  & .023 & .019 & .002 \\
    FW   & 1 &  -.005  & .017 & -.001 & -.030 \\
    \hline
    $\bar{x}$ && .006 & \textbf{.015} & .007 & -.090 \\
    $\sigma$  && .010 & .010 & .010 & .018 \\
    \hline
    \end{tabular}

\caption{
{\normalsize
Adding tree features to the best-performing single
features changes $F_1$ and $\kappa$ scores across the three data sets.
}}
\label{fig:multiple-features}
\end{center}
%}

\end{figure}


We find that annotated skeleton provides the best performance boost across all
three domains. This confirms our suspicions that structural tree information
provides the most useful information.

Other interesting results were the performance of non-unigram function words and
tree depth. Although previously shown to be a relevant feature
in \cite{acl-chen-2011}, tree depth was one of the worst performers (at least
in our data sets).

We do not believe $n$-grams of function words have seriously been considered as
a feature, but bigrams of function words worked well when combined with tree
features for the nationality and essay data sets. 

\subsection{Validity of Tree Features}

We use the correlation coefficient as described in \cite{sigir-ng-1997}
to explore the efficacy of the tree features. Looking at the highest weighted
features, we should be able to rationalize their appearance. We choose to leave
tree depth and branch factor out of these comparisons since they performed so
poorly previously.

Given the following metrics for a term $t$ and a category $c_i$ we can define
the probabilities:

\begin{enumerate}
    \item $P(t,c_i)$: presence of $t$, membership in $c_i$
    \item $P(t, \overline{c_i})$: presence of $t$, non-membership in $c_i$
    \item $P(\overline{t}, c_i)$: absence of $t$, membership in $c_i$
    \item $P(\overline{t}, \overline{c_i})$ absence of $t$, non-membership in
        $c_i$,
\end{enumerate}

Then, with $N$ total documents, the correlation coefficient (CC) can be written
as follows: \\

$CC(t, c_i) = $ \\
$$ \frac{\sqrt{N}[P(t,c_i)P(\overline{t},\overline{c_i}) - 
    P(t,\overline{c_i})P(\overline{t},c_i))]}
    {\sqrt{P(t)P(\overline{t})P(c_i)P(\overline{c_i})}} $$ \\

The correlation coefficient can be viewed as a one-sided Chi-square metric
\cite{sigkdd-zheng-2004}. That is, the features selected by CC are most
indicative of class membership only (as opposed to membership and
non-membership).

\begin{figure*}[ht]
%{\small

\begin{center}

    \textbf{$c_i$ = English} \\
    \begin{tabular}{|l|l|l|l|l|}
        \hline
        \textbf{Words} & \textbf{Skel} & \textbf{ASkel} & \textbf{RR }& \textbf{SC }\\
        \hline
        educ     & ()             &  VBG()     &  NP$\rightarrow$(NP)(VP)      &  VBG  \\
        financi  & (()()())       &  DT()      &  VP$\rightarrow$(MD)(ADVP)(VP)&  DT   \\
        individu & (()()()())     &  JJ()      &  NP$\rightarrow$(NP)(PP)      &  :    \\
        believ   & (()(()()()())) &  NNS()     &  S$\rightarrow$(VP)           &  -RRB-\\
        right    & (()())         &  NP(()()())&  PP$\rightarrow$(TO)(NP)      &  -LRB-\\
        \hline
    \end{tabular}
    
    \vspace{.1in}
    
    \textbf{$c_i$ = Chinese} \\
    \begin{tabular}{|l|l|l|l|l|}
        \hline
        \textbf{Words} & \textbf{Skel} & \textbf{ASkel} & \textbf{RR }& \textbf{SC }\\
        \hline
        china &  (()(()(()(()(()(()))))))                  & JJR()
        & ADVP$\rightarrow$(DT)       &  JJR \\
        knowledg &  (((()())(()(())))()()((()())(()(())))) &  RBR()
        &  ADVP$\rightarrow$(DT)(RBR)  &  RBR \\
        partjob &  ((())(()(())((())((())(())))))          &
        ADVP(()()(()((())(()(())))))&  NP$\rightarrow$(NP)(,)(SBAR)& \$   \\
        chines &  (()((()()()()())(()(()()))))             &
        SBAR(()((()(()(()(()))))))  &  ADJP$\rightarrow$(JJR)      &  DT  \\
        hold &  (()(()(()()))(()((()())(()(()(()()))))))   &
        VP(()(()(()(()(()(()))))))  &  NP$\rightarrow$(JJR)        &  FRAG\\
        \hline
    \end{tabular}

    \vspace{.1in}
    
    \textbf{$c_i$ = Japanese} \\
    \begin{tabular}{|l|l|l|l|l|}
        \hline
        \textbf{Words} & \textbf{Skel} & \textbf{ASkel} & \textbf{RR }& \textbf{SC }\\
        \hline
        think & (())                      &  PRP() &  NP$\rightarrow$(PRP)         &  PRP\\
        smoke &  (()(()))                 &  NP(())&  VP$\rightarrow$(VBP)(NP)     &  VBP\\
        seat &  ((())(()(()())))          &  VBP() &  NP$\rightarrow$(NN)          &  .  \\
        money &  (()(())(()(()(()())))()) &  .()   &  VP$\rightarrow$(VBP)(S)      &  LS \\
        japan &  ((())(()(()()))())       &  LS()  &  S$\rightarrow$(CC)(NP)(VP)(.)&  '' \\
        \hline
    \end{tabular}

    \caption{{\normalsize Samples of the highest ranked features for each language as selected
    by the correlation coefficient metric. Note that the words are stemmed.}}

\label{fig:validity}
\end{center}

%}
\end{figure*}


The highly-ranked word features are intuitive, especially interesting are
``china'' and ``chines'' for $c_i$ = Chinese, and ``japan'' for $c_i$ =
Japanese.

We have a few observations regarding the syntactic features:

\begin{enumerate}
    \item Somewhat surprisingly, structural tree features are significantly
        \emph{shorter} for native speakers
    \item Native speakers use parenthesis (-RRB- and -LRB-) and colons (but not
        semicolons) much more than non-native speakers do
    \item Non-native speakers more often start phrases with conjunctions
    \item Native speakers use more obscure rewrite rules such as
        VP$\rightarrow$MD, ADVP, VP
\end{enumerate}

For example, the sentence \emph{``And if there are unexpected expenses, material
for their lesson, for example, they may not be able to pay money to it only with
monthly allowance"} shows it beginning with a conjunction, and containing a
relatively complex (or convoluted) structure.

Another sentence by a native speaker \emph{``Indeed, students who have a
part-time job (like I did) quickly change their perspective"} shows a
non-standard sentence beginning and the (seemingly) popular parentheses.

These observations are intuitive and lend credibility to the tree features,
rationalizing their excellent performance when combined with simple features.

\section{Conclusions and Future Work}

We compared combinations of simple $n$-gram text representation models with new
and existing tree features. We showed that the novel structural tree features
are most effective and when combined with a simpler lexical model, capturing
multiple perspectives of the same text.

Using these new methods, we display performance gains on existing corpora
across domains. This demonstrates the generality and usefulness of our features.
We showed that the new structural features combine better with simple features
than existing tree representations such as rewrite rules and tree depth.

Additionally, the structural tree features introduced are not restricted to
probabilistic context-free grammars as mainly discussed here; they could be
applied to other tree structures as well: abstract syntax trees for source code
analysis, dependency parses for more linguistic analysis, and even HTML or XML
data for Web page or structured document comparisons.

We aimed to answer $Q_1)$ Are the new features orthogonal to the existing ones?
and $Q_2)$ Can we combine the new features with old ones to improve accuracy?
Based on our experimental results, we can answer \emph{yes} to both. We assert
the new features are orthogonal due to lack of syntactic information and
positive $F_1$ score gain after adding them to the lexical features as shown in
Figure~\ref{fig:multiple-features}. We answer $Q_2$ affirmatively with
Figure~\ref{fig:three-datasets}.

In the future we would like to explore these features in tree structures other
than PCFGs, as well in other domains such as clustering and information
retrieval.  Using structural tree features in a topic modeling context would
allow distributions of structures to be obtained for each class more easily than
with machine learning algorithms such as SVM\@. This leads to better
interpretability of features, offering clearer explanations of why some classes
favor certain structures.

We would also like to compare these new methods with the $k$-embedded-edge
subtrees discussed in \cite{sigir-kim-2011}, as well as using their proposed
feature reduction frequent tree pattern pruning. Additionally, we would be
interested in seeing how the features respond to dimensionality reduction
techniques, as the number of skeleton and annotated skeleton features is usually
quite large.

\section*{Acknowledgments}

The authors would like to thank Chase Geigle for his collaborative effort on the
toolkit used to run the experiments as well as helpful discussion regarding this
work. This material is based upon work supported in part by the National Science
Foundation under Grant Number CNS-1027965.

\newpage

\bibliographystyle{IEEEtran}
\IEEEtriggeratref{17}
\newpage
\bibliography{./bib}

\end{document}

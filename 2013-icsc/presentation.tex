% rubber: setlist arguments --shell-escape
\documentclass[handouti]{beamer}
\usepackage{verbatim}
\usepackage{multirow,bigdelim}
\usepackage{graphicx}
\usepackage{./parsetree}
\usetheme{Luebeck}
\usepackage[utf8]{inputenc}

\newcommand\MyLBrace[2]{%
      \left.\rule{0pt}{#1}\right\}\text{#2}}
\newcommand\groups{%
    $\begin{array}{l}
    \MyLBrace{5ex}{best non-tree features} \\ 
    ~ \\
    \MyLBrace{7ex}{best tree features} \\
    ~ \\
    \MyLBrace{11ex}{best combined features} 
    \end{array}$
}

\setbeamerfont{frametitle}{size={\fontsize{22}{22}}}
\setbeamerfont{title}{size={\fontsize{22}{22}}}

\ptnodefont{\footnotesize\rm}{9pt}{2pt}  % font and strut height/depth: nodes
\ptleaffont{\footnotesize\it}{9pt}{2pt}  % font and strut height/depth: leaves

\setbeamertemplate{frametitle}[default][center]

%gets rid of bottom navigation bars
\setbeamertemplate{footline}[page number]{}

%gets rid of navigation symbols
\setbeamertemplate{navigation symbols}{}

\title[Structural Parse Tree Features for Text Representation]
{Structural Parse Tree Features \\ for Text Representation}
\author{Sean Massung $\cdot$ ChengXiang Zhai $\cdot$ Julia Hockenmaier \\
{\scriptsize University of Illinois at Urbana-Champaign}}
\date{~\\~\\16th September 2013}
\institute{~\\~\\ 7$^{th}$ International Conference on Computational Semantics
    \\ Irvine, California}

\begin{document}
\frame{\titlepage}
\Large % darp

\begin{frame}
    \frametitle{Essay Scoring}
    \emph{``Computers these days have made positive changes to all of our every
        day lives. People have made better connections with family and friends,
        and even have the ability to make new ones. But with that in mind, is
        the computer truly a harmful or helpful device?''}

        \vspace{.3in}
        \begin{center}
        {\normalsize How would you score this essay on a scale of 0 to 12?}
        \end{center}
\end{frame}

\begin{frame}
    \frametitle{Movie Reviews}
    \emph{``I saw this movie last night after being coaxed to by a few
        friends of mine. I'll admit that I was reluctant to see it because from
        what I knew of Ashton Kutcher he was only able to do comedy.  I was
    wrong.''}

        \vspace{.3in}
        \begin{center}
        {\normalsize Was the opinion of this movie positive or negative?}
        \end{center}
\end{frame}

\begin{frame}
    \frametitle{Nationality Detection}
    \emph{``I agree this opinion that part-time job is important for college
    school students, because I studied a lot of thing with my part-time job.
    At first, the communication skill is necessary to work.''}

        \vspace{.3in}
        \begin{center}
        {\normalsize Was this essay written by a Chinese, Japanese, or English
        student?}
        \end{center}
\end{frame}

\begin{frame}
    \frametitle{Text Representation}
    \begin{itemize}
        \itemsep1.5em
        \item[] Text representation is a fundamental issue; we can represent any
            text by a set of extracted features

        \vspace{.5in}
        {\normalsize
        \begin{tabular}{ccl}
            & $\rightarrow$ & \emph{clustering} \\
            \emph{document} $\rightarrow$ \emph{feature vector} & $\rightarrow$
            & \emph{classification} \\
            & $\rightarrow$ & \emph{information retrieval} \\
            & $\rightarrow$ & \emph{data mining} \\
            & $\rightarrow$ & \emph{recommendation} \\
            & $\rightarrow$ & $\ldots$\\
        \end{tabular}}

        \vspace{.3in}
        \begin{center}
        It all starts with a feature vector!
        \end{center}
    \end{itemize}
\end{frame}

\begin{frame}[fragile]
    \frametitle{Standard Features}
    \begin{itemize}
        \itemsep1.5em
        \item Word $n$-grams {\normalsize (Furnkranz 1998) }
                \begin{center}
                {\normalsize
                $\{ they:1,  have:1,  mani:1,  theoret:1,  idea:1  \}$
                }
                \end{center}
            \item Function words {\normalsize (Stamatatos 2009)}
                \begin{center}
                {\normalsize
                    $\{ they:1, have:1, mani:1 \}$
                }
                \end{center}
            \item POS tag $n$-grams {\normalsize (Argamon et al, 1998)}
                \begin{center}
                {\normalsize
                    $\{ PRP:1, VBP:1, JJ:2, NNS:1, .:1 \}$
                }
                \end{center}
     \end{itemize}
\end{frame}

\begin{frame}
    \frametitle{Nationality Detection}

    {\small
    $<s>$ PRP VBP DT NN WDT JJ NN VBZ JJ IN NN NN NNS , IN PRP VBD DT NN IN NN IN
    PRP\$ JJ NN . $</s>$

    \vspace{.3in}
    $<s>$ IN RB , DT NN NN VBZ JJ TO VB . $</s>$

    \vspace{.3in}
    $<s>$ PRP MD VB TO NN CC NN , IN PRP VBZ RB RB JJ TO VB TO NNS IN NN NN .
    $</s>$
    }

    \vspace{.3in}
    \begin{center}
    {\normalsize Was this essay written by a Chinese, Japanese, or English
    student?}
    \end{center}
\end{frame}

\begin{frame}[fragile]
    \frametitle{Features from Parse Trees}
    {\small
    \begin{tabular}{cc}
        \emph{ Example parse tree of a sentence } 
        &
        \emph{ ~~~ Rewrite rule features} \\
    \end{tabular}
     \begin{columns}[c]
        \begin{column}{.5\textwidth}
            \begin{parsetree}
              (.S. (.NP. (.PRP. `They')) (.VP. (.VBP. `have')
              (.NP. (.JJ. `many') (.JJ. `theoretical') (.NNS. `ideas')))
              (.$\bullet$. `$\bullet$') )
            \end{parsetree}
            \vfill
        \end{column}
        \begin{column}{.5\textwidth}
            \begin{tabular}{ll}
                \begin{parsetree}
                    (.S. .NP. .VP. .$\bullet$.)
                \end{parsetree} &
                \begin{parsetree}
                    (.NP. .PRP.)
                \end{parsetree} \\
                ~ & ~ \\ % add space
                \begin{parsetree}
                    (.VP. .VBP. .NP.)
                \end{parsetree} &
                \begin{parsetree}
                    (.NP. .JJ. .JJ. .NNS.)
                \end{parsetree} \\
            \end{tabular}
        \end{column}
    \end{columns}
    }
\end{frame}
\begin{frame}[fragile]
    \frametitle{Standard Tree Features}
    \begin{itemize}
        \itemsep1em
        \item Syntactic categories {\normalsize (Stamatatos 2009)}
    
            \begin{center}
            {\small
                \{ $S:1, NP:2, PRP:1, VP:1, VBP:1, JJ:2, NNS:1$ \}
            }
            \end{center}
        \item Rewrite rules {\normalsize (Kim 2011)}

             \begin{center}
            {\small
                $\{ S\rightarrow NP,VP,.$:$1, NP \rightarrow PRP$:$1,$ \\$
                VP\rightarrow VBP,NP$:$1, NP \rightarrow JJ,JJ,NNS$:$1 \}$
            }
            \end{center}
            \item Branching factors {\normalsize (Chen 2011)}
            
            \begin{center}
            {\small
                \{ 1:7, 2:1, 3:2 \}
            }
            \end{center}
    \item Tree depth {\normalsize (Chen 2011)}
            
    %%%%%%  \begin{center}
    %%%%%%  {\small
    %%%%%%      \{ 4:1 \}
    %%%%%%  }
    %%%%%%  \end{center}

        \item $k$-ee trees {\normalsize (Kim 2011)}
    \end{itemize}
\end{frame}

%%  \begin{frame}
%%      \frametitle{Existing Feature: $k$-ee Trees {\normalsize (Kim 2011)}}
%%      \begin{itemize}
%%          \itemsep1.3em
%%          \item 0-ee tree: simple rewrite rules
%%          \item $k$-ee tree: $k$ connected ancestor subtrees
%%          \item Frequent pattern mining to find relevent features in huge feature
%%              space
%%          \item Overfitting? Computationaly feasible?
%%      \end{itemize}
%%  \end{frame}

\begin{frame}
    \frametitle{Nationality Detection}

    {\small
    (S(NP(PRP))(VP(VBP)(NP(DT)(NN))(SBAR(IN)(S(NP(JJ)(NN))
    (VP(VBZ)(ADJP(JJ)(PP(IN)(NP(NN)(NN)(NNS)))))))(,)(SBAR(IN)
    (S(NP(PRP))(VP(VBD)(NP(NP(DT)(NN))(PP(IN)(NP(NN))))
    (PP(IN)(NP(PRP\$)(JJ)(NN)))))))(.))

    \vspace{.3in}
    (S(PP(IN)(NP(JJ)))(,)(NP(DT)(NN)(NN))(VP(VBZ)(ADJP(JJ)
    (S(VP(TO)(VP(VB))))))(.))

    \vspace{.3in}
    (S(NP(PRP))(VP(MD)(VP(VB)(PP(TO)(NP(NN)(CC)(NN)))(,)(SBAR(IN)
    (S(NP(PRP))(VP(VBZ)(RB)(ADJP(RB)(JJ)(S(VP(TO)(VP(VB)(PP(TO)
    (NP(NP(NNS))(PP(IN)(NP(NN)(NN))))))))))))))(.))
    }

        \vspace{.3in}
        \begin{center}
        {\normalsize Was this essay written by a Chinese, Japanese, or English
        student?}
        \end{center}
\end{frame}

\begin{frame}
    \frametitle{Motivation}
    \begin{itemize}
        \itemsep1.5em
        \item Simple lexical features + deep syntactic features = best
            performance {\normalsize (Stamatatos, 2009)}
        \item Can we generalize the ``deep syntactic features'' even more?
        \item Making a more general feature would help with overfitting and data
            sparseness
    \end{itemize}
\end{frame}

\begin{frame}
    \frametitle{This Paper}
    \begin{itemize}
        \itemsep1.5em
        \item Investigate \textbf{new dimension} of text representation based on
            parse trees with emphasis on structure
        \item Define features solely based on \textbf{structural patterns} by
            ignoring all or almost all of the syntactic categories
        \item Evaluate \textbf{skeleton-based features} using three different
            categorization tasks
    \end{itemize}
\end{frame}

\begin{frame}
    \frametitle{New Structural Features}
    \begin{itemize}
        \itemsep1.5em
        \item \textbf{Skeleton}: ignore all syntactic categories in parse trees
        \item \textbf{Annotated Skeleton}: ignore all syntactic categories
            \emph{except} the root
    \end{itemize}
\end{frame}

\begin{frame}[fragile]
    \frametitle{Skeleton}
    \begin{center}
    \begin{tabular}{cc}
        \begin{parsetree}
            (.x. (.x. .x.) (.x. .x. (.x. .x. .x. .x.)) .x.)
        \end{parsetree} &
        \begin{parsetree}
            (.x. .x. (.x. .x. .x. .x.))
        \end{parsetree} \\
        ~ & ~ \\
        \begin{parsetree}
            (.x. .x.)
        \end{parsetree} &
        \begin{parsetree}
            (.x. .x. .x. .x.)
        \end{parsetree} \\
    \end{tabular}

    \vspace{.2in}
    {\normalsize
    No syntactic categories or part-of-speech tags are retained}
    \end{center}
\end{frame}

\begin{frame}[fragile]
    \frametitle{Annotated Skeleton}
    \begin{center}
    \begin{tabular}{ccc}
        \begin{parsetree}
            (.S. (.x. .x.)(.x. .x.(.x. .x. .x. .x.)) .x. )
        \end{parsetree} &
        \begin{parsetree}
            (.VP. .x. (.x. .x. .x. .x.))
        \end{parsetree} \\
        ~ & ~ \\
        \begin{parsetree}
            (.NP. .x.)
        \end{parsetree} &
        \begin{parsetree}
            (.NP. .x. .x. .x.)
        \end{parsetree} \\
    \end{tabular}

    \vspace{.2in}
    {\normalsize
    Only the topmost syntactic category is retained}
    \end{center}
\end{frame}

\begin{frame}
    \frametitle{Questions to Answer}
    \emph{Are the new features orthogonal to existing ones?}

    \vspace{.5in}
    \emph{Can we combine the new features with existing ones to improve
    classification accuracy?}
\end{frame}

\begin{frame}
    \frametitle{Data Sets}
    \begin{itemize}
        \itemsep1.5em
        \item Foundation (2012) data set is scored student essays on a range of
            0 to 12
        \item Maas et al. (2011) consists of movie reviews from the
            International Movie Database
        \item Ishikawa (2009) data set consists of essays written in English by
            native Chinese, Japanese, and English students
    \end{itemize}
\end{frame}

\begin{frame}
    \frametitle{Methodology}
    \begin{itemize}
        \itemsep1.5em
        \item Features generated with the Stanford Parser
        \item Compare lexical and semantic single features (with varying $n$
            where applicable)
        \item Compare all lexical + semantic pairings for combined feature sets
        \item SVM classifier (liblinear) with five-fold cross-validation
    %%  \item $F_1$ and accuracy for nationality detection and quadratic
    %%      weighted kappa scoring ($\kappa$) for the essay grades
    %%  \item Correlation coefficient to explore feature validity
    \end{itemize}
\end{frame}

%\begin{figure*}[ht!]
\begin{center}
%{\small

\begin{tabular}{ccc}

    \textbf{Nationality Detection} &
    \textbf{Essay Scoring} &
    \textbf{Sentiment Analysis} \\

    % nationality
    \begin{tabular}{|lccc|}
    \hline
    \emph{Method} & $n_{best}$ & $F_1$ & $A$ \\
    \hline
    \hline
    Word    & 1 & .827 & .916 \\
    POS     & 2 & .810 & .902 \\
    FW      & 2 & .728 & .876 \\
    \hline
    \hline
    RR      &   & .778 & .844 \\
    Skel    &   & .510 & .806 \\
    ASkel   &   & .721 & .870 \\
    SC      &   & .703 & .844 \\
    Branch  &   & .508 & .786 \\
    Depth   &   & .436 & .774 \\
    \hline
    \hline
    \textbf{ASkel + Word} & \textbf{1}  & \textbf{.885} & \textbf{.942} \\
    SC + Word    & 1  & .867 & .936 \\
    RR + Word    & 1  & .854 & .924 \\
    ASkel + POS  & 2  & .811 & .900 \\
    RR + POS     & 2  & .810 & .896 \\
    Skel + Word  & 1  & .809 & .912 \\
    ASkel + FW   & 2  & .799 & .904 \\
    RR + FW      & 2  & .795 & .898 \\
    \hline
    \end{tabular}
    &
    % essay
    \begin{tabular}{|lcc|}
    \hline
    \emph{Method} & $n_{best}$ & $\kappa$ \\
    \hline
    \hline
    \textbf{Word}   & \textbf{2} & \textbf{.889} \\
    POS    & 2 & .765 \\
    FW     & 1 & .845 \\
    \hline
    \hline
    RR     &   & .702 \\
    Skel   &   & .356 \\
    ASkel  &   & .658 \\
    SC     &   & .431 \\
    Branch &   & -.087 \\
    Depth  &   & -.098 \\
    \hline
    \hline
    SC + Word    & 2 & .834 \\
    ASkel + FW   & 1 & .822 \\
    RR + Word    & 2 & .807 \\
    ASkel + Word & 2 & .791 \\
    RR + FW      & 1 & .786 \\
    ASkel + POS  & 2 & .782 \\
    Skel + POS   & 2 & .768 \\
    RR + POS     & 2 & .761 \\
    \hline
    \end{tabular}
    &
    % sentiment
    \begin{tabular}{|lccc|}
    \hline
    \emph{Method} & $n_{best}$ & $F_1$ & $A$ \\
    \hline
    \hline
    Word   & 1 & .820 & .820 \\
    POS    & 3 & .662 & .662 \\
    FW     & 1 & .687 & .687 \\
    \hline
    \hline
    RR     &   & .650 & .650 \\
    Skel   &   & .556 & .557 \\
    ASkel  &   & .654 & .654 \\
    SC     &   & .555 & .568 \\
    Branch &   & .493 & .499 \\
    Depth  &   & .533 & .535 \\
    \hline
    \hline
    \textbf{Skel + Word}  & \textbf{1 } & \textbf{.828} & \textbf{.828} \\
    RR + Word    & 1  & .824 & .824 \\
    ASkel + Word & 1  & .824 & .824 \\
    SC + Word    & 1  & .822 & .822 \\
    ASkel + FW   & 1  & .704 & .704 \\
    RR + FW      & 1  & .686 & .686 \\
    ASkel + POS  & 3  & .685 & .685 \\
    Skel + FW    & 1  & .682 & .682 \\
    \hline
    \end{tabular}

\end{tabular}
\caption{
{\normalsize Comparison of single and combined features across data sets.  RR, Skel,
    ASkel, and SC refer to Rewrite Rules, Skeleton, Annotated Skeleton, and
    Syntactic Category respectively.
}}
\label{fig:three-datasets}
%}

\end{center}
\end{figure*}



\begin{frame}
    \frametitle{Essay Scoring}
    \begin{itemize}
        \itemsep1.5em
        \item Standard features: \textbf{bigram words} performed the best
            ($\kappa=0.889$)

        \item New and existing tree features: \textbf{rewrite rules} won with
            $\kappa=0.702$

        \item Of all combined features, \textbf{syntactic categories + unigram
            words} performed best overall with $\kappa=0.834$
    \end{itemize}
\end{frame}

\begin{frame}
    \frametitle{Poor Performance}
    \begin{itemize}
        \itemsep1.5em
        \item Objective \emph{vs} subjective
        \item Content \emph{vs} style
        \item Average of three \emph{human} ratings
        \item Negative $\kappa$ scores for tree depth and branching factor
    \end{itemize}
\end{frame}

\begin{frame}
    \frametitle{Sentiment Analysis}
    \begin{itemize}
        \itemsep1.5em
        \item Standard features: \textbf{unigram words} performed the best
            ($82\%$ accuracy)

        \item New and existing tree features: \textbf{rewrite rules} won with
            $65\%$ accuracy

        \item Of all combined features, \textbf{annotated skeleton + unigram
            words} performed best overall with $82.8\%$ accuracy
    \end{itemize}
\end{frame}

\begin{frame}
    \frametitle{Nationality Detection}
    \begin{itemize}
        \itemsep1.5em
        \item Standard features: \textbf{unigram words} performed the best
            ($91.6\%$ accuracy)

        \item New and existing tree features: \textbf{rewrite rules} won with
            $84.4\%$ accuracy

        \item Of all combined features, \textbf{annotated skeleton + unigram
            words} performed best overall with $94.2\%$ accuracy
    \end{itemize}
\end{frame}

\begin{frame}
    \frametitle{The Correlation Coefficient}

    \begin{itemize}
        \item We want to justify that the top-ranked features are intuitively
            useful
        \item For each (class, term) pair, we calculate the correlation
            coefficient:
    \end{itemize}

    \vspace{.2in}

$$CC(t, c_i) = \frac{P(t,c_i)P(\overline{t},\overline{c_i}) - 
    P(t,\overline{c_i})P(\overline{t},c_i)}
    {\sqrt{P(t)P(\overline{t})P(c_i)P(\overline{c_i})}} $$ \\
\end{frame}

\begin{frame}
    \frametitle{Best Features via Correlation Coefficient}
    \begin{center}
    {\tiny

    \textbf{English}

    \begin{tabular}{|l|l|l|l|l|}
        \hline
        \textbf{Words} & \textbf{Skel} & \textbf{ASkel} & \textbf{RR }& \textbf{SC }\\
        \hline
        educ     & ()             &  VBG()     &  NP$\rightarrow$(NP)(VP)      &  VBG  \\
        financi  & (()()())       &  DT()      &  VP$\rightarrow$(MD)(ADVP)(VP)&  DT   \\
        individu & (()()()())     &  JJ()      &  NP$\rightarrow$(NP)(PP)      &  :    \\
        believ   & (()(()()()())) &  NNS()     &  S$\rightarrow$(VP)           &  -RRB-\\
        right    & (()())         &  NP(()()())&  PP$\rightarrow$(TO)(NP)      &  -LRB-\\
        \hline
    \end{tabular}

    \vspace{.2in}
    \textbf{Chinese}

    \begin{tabular}{|l|l|l|l|l|}
        \hline
        \textbf{Words} & \textbf{Skel} & \textbf{ASkel} & \textbf{RR }& \textbf{SC }\\
        \hline
        china &  (()(()(()(()(()(()))))))                  & JJR()
        & ADVP$\rightarrow$(DT)       &  JJR \\
        knowledg &  (((()())(()(())))()()((()())(()(())))) &  RBR()
        &  ADVP$\rightarrow$(DT)(RBR)  &  RBR \\
        partjob &  ((())(()(())((())((())(())))))          &
        ADVP(()()(()((())(()(())))))&  NP$\rightarrow$(NP)(,)(SBAR)& \$   \\
        chines &  (()((()()()()())(()(()()))))             &
        SBAR(()((()(()(()(()))))))  &  ADJP$\rightarrow$(JJR)      &  DT  \\
        hold &  (()(()(()()))(()((()())(()(()(()()))))))   &
        VP(()(()(()(()(()(()))))))  &  NP$\rightarrow$(JJR)        &  FRAG\\
        \hline
    \end{tabular}

    \vspace{.2in}
    \textbf{Japanese}

    \begin{tabular}{|l|l|l|l|l|}
        \hline
        \textbf{Words} & \textbf{Skel} & \textbf{ASkel} & \textbf{RR }& \textbf{SC }\\
        \hline
        think & (())                      &  PRP() &  NP$\rightarrow$(PRP)         &  PRP\\
        smoke &  (()(()))                 &  NP(())&  VP$\rightarrow$(VBP)(NP)     &  VBP\\
        seat &  ((())(()(()())))          &  VBP() &  NP$\rightarrow$(NN)          &  .  \\
        money &  (()(())(()(()(()())))()) &  .()   &  VP$\rightarrow$(VBP)(S)      &  LS \\
        japan &  ((())(()(()()))())       &  LS()  &  S$\rightarrow$(CC)(NP)(VP)(.)&  '' \\
        \hline
    \end{tabular}

}
\end{center}
\end{frame}

\begin{frame}
    \frametitle{Best Features via Correlation Coefficient}
    {\tiny
    \begin{center}
    \textbf{English}

    \begin{tabular}{|l|l|l|}
        \hline
        \textbf{Skel} & \textbf{RR }& \textbf{SC }\\
        \hline
        ()             &  NP$\rightarrow$(NP)(VP)      &  VBG  \\
        (()()())       &  VP$\rightarrow$(MD)(ADVP)(VP)&  \textbf{DT}   \\
        (()()()())     &  NP$\rightarrow$(NP)(PP)      &  :    \\
        (()(()()()())) &  S$\rightarrow$(VP)           &  -RRB-\\
        (()())         &  PP$\rightarrow$(TO)(NP)      &  -LRB-\\
        \hline
    \end{tabular}

    \vspace{.1in}
    \textbf{Chinese}

    \begin{tabular}{|l|l|l|}
        \hline
        \textbf{Skel} & \textbf{RR }& \textbf{SC }\\
        \hline
        (()(()(()(()(()(())))))) & ADVP$\rightarrow$(DT)       &  JJR \\
         (((()())(()(())))()()((()())(()(())))) & ADVP$\rightarrow$(DT)(RBR)  &  RBR \\
        ((())(()(())((())((())(())))))  & NP$\rightarrow$(NP)(,)(SBAR)& \$   \\
        (()((()()()()())(()(()())))) &  ADJP$\rightarrow$(JJR)      &
        \textbf{DT}  \\
        (()(()(()()))(()((()())(()(()(()()))))))   & NP$\rightarrow$(JJR)        &  FRAG\\
        \hline
    \end{tabular}

    \vspace{.1in}
    \textbf{Japanese}

    \begin{tabular}{|l|l|l|}
        \hline
        \textbf{Skel} & \textbf{RR }& \textbf{SC }\\
        \hline
         (())                      & NP$\rightarrow$(PRP)         &  PRP\\
          (()(()))                 & VP$\rightarrow$(VBP)(NP)     &  VBP\\
         ((())(()(()())))          & NP$\rightarrow$(NN)          &  .  \\
          (()(())(()(()(()())))()) & VP$\rightarrow$(VBP)(S)      &  LS \\
        ((())(()(()()))())       & \textbf{S$\rightarrow$(CC)(NP)(VP)(.)}&  '' \\
        \hline
    \end{tabular}
    \end{center}}

    {\normalsize
    \emph{``And if there are unexpected expenses, material for their lesson, for
    example, they may not be able to pay money to it only with monthly
    allowance."}}
\end{frame}

\begin{frame}
    \frametitle{Best Features via Correlation Coefficient}
    {\tiny
    \begin{center}
    \textbf{English}

    \begin{tabular}{|l|l|l|}
        \hline
        \textbf{Skel} & \textbf{RR }& \textbf{SC }\\
        \hline
        ()             &  NP$\rightarrow$(NP)(VP)      &  VBG  \\
        (()()())       &  VP$\rightarrow$(MD)(ADVP)(VP)&  \textbf{DT}   \\
        (()()()())     &  NP$\rightarrow$(NP)(PP)      &  :    \\
        (()(()()()())) &  S$\rightarrow$(VP)           &  \textbf{-RRB-}\\
        \textbf{(()())}         &  PP$\rightarrow$(TO)(NP)      &  \textbf{-LRB-}\\
        \hline
    \end{tabular}

    \vspace{.1in}
    \textbf{Chinese}

    \begin{tabular}{|l|l|l|}
        \hline
        \textbf{Skel} & \textbf{RR }& \textbf{SC }\\
        \hline
        (()(()(()(()(()(())))))) & ADVP$\rightarrow$(DT)       &  JJR \\
         (((()())(()(())))()()((()())(()(())))) & ADVP$\rightarrow$(DT)(RBR)  &  RBR \\
        ((())(()(())((())((())(())))))  & NP$\rightarrow$(NP)(,)(SBAR)& \$   \\
        (()((()()()()())(()(()())))) &  ADJP$\rightarrow$(JJR)      &  DT  \\
        (()(()(()()))(()((()())(()(()(()()))))))   & NP$\rightarrow$(JJR)        &  FRAG\\
        \hline
    \end{tabular}

    \vspace{.1in}
    \textbf{Japanese}

    \begin{tabular}{|l|l|l|}
        \hline
        \textbf{Skel} & \textbf{RR }& \textbf{SC }\\
        \hline
         (())                      & NP$\rightarrow$(PRP)         &  PRP\\
          (()(()))                 & VP$\rightarrow$(VBP)(NP)     &  VBP\\
         ((())(()(()())))          & NP$\rightarrow$(NN)          &  .  \\
          (()(())(()(()(()())))()) & VP$\rightarrow$(VBP)(S)      &  LS \\
          ((())(()(()()))())       & S$\rightarrow$(CC)(NP)(VP)(.)&  '' \\
        \hline
    \end{tabular}
    \end{center}}

    {\normalsize
    \emph{``Indeed, students who have a part-time job (like I did) quickly
change their perspective."}}
\end{frame}


\begin{frame}
    \frametitle{Questions to Answer}
    \emph{Are the new features orthogonal to existing ones?}

    \vspace{.5in}
    \emph{Can we combine the new features with existing ones to improve
    classification accuracy?}
\end{frame}

\begin{frame}
    \frametitle{Concluding Remarks}
        \begin{itemize}
        \itemsep1.5em
        \item We compared simple $n$-gram models with new and existing tree
            features
        \item New structural features combine better with simple features
        \item Structural features aren't restricted to PCFGs 
    \end{itemize}
\end{frame}

\begin{frame}
    \frametitle{Future Work}
    \begin{itemize}
        \itemsep1.5em
        \item Explore other tree structures besides PCFGs
        \item Use in clustering and IR
        \item Topic modeling
        \item Investigate dimensionality reduction
    \end{itemize}
\end{frame}

\begin{frame}
    \frametitle{Thank you!}
    \begin{center}
    Questions? Comments?
    \end{center}
\end{frame}

\end{document}

% !TEX root = paper.tex
% !TEX encoding = UTF-8 Unicode  

\subsection{Finnish tokenization using Morfessor and word-lattices}
\label{sec:fien1}

%\begin{table*}[!h]
    {\small
    \begin{center}
        \begin{tabular}{lcclllrr}
            \hline
            \textbf{Method} & \textbf{LM} & \textbf{Phrase} & \textbf{Train} & \textbf{Tune} &
            \textbf{Test} & \textbf{BLEU-c} & \textbf{BLEU} \\
            \hline
            Baseline & 5 & 5 & Europarl & Newsdev & Newstest 2015 &
            \textbf{15.09} & \textbf{16.95} \\
            Morfessor $\alpha=0.5$ & 5 & 8 & Europarl & Newsdev & Newstest 2015
            & 14.88 & 15.67 \\
            Hierarchical phrase-based & 6 & 5 & Europarl & Newsdev &
            Newstest 2015 & 14.45 & 14.99 \\
            Word lattice ($n=2$) & 6 & 8 & Europarl & Newsdev & Newstest 2015
            & 14.00 & 14.67 \\
            Word lattice ($n=5$) & 6 & 8 & Europarl & Newsdev & Newstest 2015
            & 13.95 & 14.68 \\
            \hline
            Baseline-omorfi & 5 & 5 & Europarl & Newsdev & Newstest 2015 & 15.25 & 16.14 \\
            V1-omorfi & 5 & 5 & Europarl & Newsdev & Newstest 2015 & 14.00 & 14.79 \\
            V2-omorfi & 5 & 5 & Europarl & Newsdev & Newstest 2015 & 14.32 & 15.14 \\
            V3-omorfi & 5 & 5 & Europarl & Newsdev & Newstest 2015 & \textbf{15.98} & \textbf{16.90} \\
            \hline
        \end{tabular}
    \caption{Results from the testing stage with Newstest2015 testing data.}
    \label{table:test}
    \end{center}
    }
\end{table*}



We begin by adapting the lattice technique of \newcite{Dyer:2009:StatMT} to Finnish. 
%
We train a standard phrase-based machine translation model on a new corpus: 
%
on the source side we concatenate the original data with its one-best segmentation according to a Morfessor~\cite{creutz} model trained on the original data, and on the target side we simply concatenate it with itself. 
%
The result is a corpus that is twice as long as the original data, but that aligns both segmented and unsegmented Finnish sentences with their English counterparts.
%
This ensures that we will have phrases in our phrase table that correspond with both the original unsegmented words as well as for individual morphemes.

At tuning and test time, we then decompose our input into a word lattice input that reflects the uncertainty of the decomposition of each word in the sentence~\cite{Dyer:2008:ACL-HLT}.
%
We construct the lattice by considering the top five best segmentations for each word according to our Morfessor model.
%
The start and end of each word in the original sentence is a node, and we place edges and nodes between the two such that the edge is labeled with a string output and its target is a node that represents the partial output of the word thus far.
%
Each of the edges is also weighted with a certain probability, reflecting the likelihood of using that specific edge, given that we are at a specific node. 

\begin{figure}[!t]
  \begin{center}
%  \begin{sideways}
    \begin{tikzpicture}
      \tikzstyle{main}=[circle, minimum size = 1mm, inner sep=0.5mm, thick, draw =black!80, node distance = 15mm]
%      \tikzstyle{observed}=[circle, minimum size = 10mm, thick, draw=black!80, node distance = 10mm, fill = black!10]
%      \tikzstyle{invismain}=[circle, minimum size = 10mm, thick, draw =white!100, node distance = 10mm]
      \tikzstyle{connect}=[-latex, thick]

      \node[main] (1) [] {1};
      \node[main] (2) [right of=1] {2};
      \node[main] (3) [above right of=1] {3};
      \node[main] (4) [right of=3] {4};
      \node[main] (5) [right of=4] {5};
      \node[main] (6) [above=15mm of 4] {6};
      \node[main] (7) [right of=5] {7};

      \path (1) edge [connect] node[below] {vilp} (2)
            (2) edge [bend right, connect] node[below=1mm] {itt\"om\"an} (7);

      \path (1) edge [bend left, connect] node[left] {vilpit} (3);
      \path (3) edge [bend left=50, connect] node[above] {t\"om\"an} (7);

      \path (3) edge [bend left, connect] node[left] {t\"om\"a} (6);
      \path (6) edge [bend left, connect] node[above] {n} (7);

      \path (4) edge [connect] node[below] {\"om\"a} (5);
      \path (5) edge [connect] node[below] {n} (7);

      \path (3) edge [connect] node[below] {t} (4);
      \path (4) edge [bend left, connect] node[above] {\"om\"an} (7);
    \end{tikzpicture}
%    \end{sideways}
    \caption{A word lattice that represents the
    top five segmentations for the Finnish word \emph{vilpitt\"om\"an}.
%    Omitted here are the weights associated with the edges.
    }
    \label{fig:wordlattice}
  \end{center}
\end{figure}

We calculate edge probabilities as follows.
%
Let $p(v | u, \Theta)$ be the probability of going to node $v$  given that we are at node $u$ under the trained Morfessor model $\Theta$ (we only concern ourselves with the case where $v$ is an adjacent to $u$).
%
Let $\mathbf{s}$ be a segmentation for the current word, represented as a set of edges $(n_1, n_2)$ through the graph.
%
Then, we set
\[
p(v \mid u, \Theta) = \frac{\sum_{\mathbf{s} : (u, v) \in \mathbf{s}}
p(\mathbf{s} \mid \Theta)}{\sum_{\mathbf{s'} : (u, v') \in \mathbf{s'}}
p(\mathbf{s'} \mid \Theta)},
\]
where the numerator is a summation of the Morfessor segmentation probabilities for segmentations that use the edge $(u, v)$, and the denominator is a summation of the Morfessor segmentation probabilities for all segmentations that pass through node $u$.

However, Morfessor gives us log likelihood scores for its segmentations.
%
Call these $\ell_{\mathbf{s}}$.
%
We then compute the following, in order to avoid roundoff errors as much as possible:
\[
p(v \mid u, \Theta) = \frac{\sum_{\mathbf{s} : (u, v) \in \mathbf{s}}
2^{\ell_\mathbf{s} - \ell_{max}}}{\sum_{\mathbf{s'} : (u, v') \in
\mathbf{s'}} 2^{\ell_{\mathbf{s'}} - \ell_{max}}},
\]
where $\ell_{max}$ is the highest log likelihood segmentation for the current word.
%
This can be seen as simply multiplying the numerator and denominator by the fixed constant $2^{-\ell_{max}}$.
%
The code for performing this lattice generation is freely available online.%
%
\footnote{\texttt{https://github.com/smassung/uiuc-wmt15}
\texttt{/tree/master/chase}}
%
We use a Morfessor model trained on the Finnish side of the Europarl parallel training data with $\alpha = 0.5$.

Table~\ref{tab:fien1} shows the output of our systems on the testing data from WMT 2015.
%
We report the scores that were obtained from Moses evaluation scripts using multi-BLEU; the numbers in the shared task are slightly different as they use the NIST BLEU scripts. 
%
Our baseline is a phrase-based default Moses configuration with the 5-gram language model, and we found this outperformed a hierarchical phrase based configuration with the same maximum phrase length and a 6-gram language model.
%
Among the segmentation methods, using a single one-best segmentation with Morfessor performed the best --- the word lattice method had disappointing performance using either the top five or top two best segmentations for the lattice generation.
%
We were unable to combine the word lattice and hierarchical phrase-based approaches together as Moses does not yet support these two features at the same time.

\begin{table}[!t]
\center 
{
	\begin{tabular}{|l|l|l|l|l|}
	\hline
	System                 & LM & TM      & BLEU             & -cased          \\ \hline
    Baseline               & 5  & 5       & \textbf{16.95}   & \textbf{15.09}  \\ \hline
    Morfessor              & 5  & 8       &         15.67    &         14.88   \\ \hline
    Hiero                  & 6  & 5       &         14.99    &         14.45   \\ \hline
    Lattice ($n=2$)        & 6  & 8       &         14.67    &         14.00   \\ \hline
    Lattice ($n=5$)        & 6  & 8       &         14.68    &         13.95   \\ \hline
	\end{tabular}
}
\caption{Results for Finnish-English (\S\ref{sec:fien1}).}
\label{tab:fien1}
\end{table}



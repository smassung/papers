% !TEX root = paper.tex
% !TEX encoding = UTF-8 Unicode

For Czech-English, we train baseline phrase-based systems with no special handling of Czech morphology.
%
We also consider experimental variants in which Czech words are morphologically segmented.
%
We use Morphessor \cite{creutz} for morphological segmentation.

Finally, we consider a re-ranking technique based on the degree of commonality between parts-of-speech (POS) in each source sentence and each respective translation of that source sentence.
%
To this end, we use MorphoDiTa \cite{morphodita} and the Stanford CoreNLP toolkit \cite{manning-EtAl:2014:P14-5} to POS tag the Czech and English sentences, respectively.
%
We next construct a dictionary that maps POS tags from one language to tags in the other.
%
After translating with Moses, each English translation in the $n$-best list is augmented with a POS intersection score, and rerank taking this new score into account.
%
We define the POS intersection score as simply the number of identical POS tags between a Czech sentence and the hypothesized English translation.

\begin{table}[!h]
\begin{center}
	\begin{tabular}{| p{4cm} | l | l |}
	\hline
	System & BLEU & BLEU-c \\ \hline
	Moses trained on Europarl & 18.59 & 17.72 \\ \hline
	Moses trained on Europarl, Common Crawl and News Commentary & 20.69 & 19.83 \\ \hline
	Stemming as pre-processing, Moses trained on Europarl & 17.88 & 17.08 \\ \hline
	Morfessor trained on Europarl, Moses trained on Europarl & 16.48 & 15.74 \\ \hline
	POS intersection, Moses trained on Europarl & 15.68 & 13.46 \\ \hline
	Morfessor trained on Europarl, POS intersection, Moses trained on Europarl & 13.43 & 13.74 \\
	\hline
	\end{tabular}
\end{center}
\caption{Results for Czech into English.}
\label{tab:ende}
\end{table}
